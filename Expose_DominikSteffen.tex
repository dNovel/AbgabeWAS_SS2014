% !!!SUCHE NACH IMPORTANT ZUM CHECKEN!!!

\documentclass[pagesize, paper=a4, fontsize=12pt,titlepage=true, headings=small, headnosepline, abstractoff, liststotoc, nochapterprefix, plainheadsepline]{scrreprt}
%\documentclass[pagesize, paper=a4, fontsize=12pt,titlepage=true, headings=small, headnosepline, abstractoff, liststotoc, nochapterprefix, plainheadsepline, twoside]{scrreprt}
\usepackage[a4paper, left=40mm, right=30mm, top=20mm, bottom=30mm]{geometry}
\usepackage[utf8]{inputenc}
\usepackage[ngerman]{babel}
\usepackage{amsmath}
\usepackage{amsfonts}
\usepackage{amssymb}
\usepackage{makeidx}
\usepackage{setspace}
\usepackage{color}
\usepackage{cite} % Paket fuer die Zitation
% \usepackage{natbib} % Erweitertes paket für Zitate.
%\usepackage{sourcesanspro}
\usepackage[T1]{fontenc}
\usepackage{lmodern}
% Bilder Settings
\usepackage{graphicx}
\usepackage [singlelinecheck=false] {caption}
\usepackage{subcaption}
\usepackage{url}
\usepackage{scrpage2}
\usepackage [singlelinecheck=false] {caption}
\usepackage{pdfpages}

% Paket fuer das anzeigen von Sourcecode
\usepackage{listings}
% Setze die Programmiersprache auf CSharp
\lstset{language=[Sharp]C} 

% Festlegung Art der Zitierung - Havardmethode: Abkuerzung Autor + Jahr
\bibliographystyle{alphadin}
%plain

% Festlegen der Sprache
\selectlanguage{ngerman}

% Settings fuer den Sourcecode START
\definecolor{mywhite}{rgb}{1,1,1}
\definecolor{mygreen}{rgb}{0,0.4,0}
\definecolor{mygray}{rgb}{0.5,0.5,0.5}
\definecolor{mykeywordgray}{rgb}{0.2,0.2,0.2}
\definecolor{mymauve}{rgb}{0.58,0,0.82}
\definecolor{bggray}{rgb}{0.97,0.97,0.97}
\definecolor{titlegray}{rgb}{0.4,0.4,0.4}

% Farbe für die Überschriften
\addtokomafont{sectioning}{\color{titlegray}\rmfamily}


\lstset{
backgroundcolor=\color{mywhite},  % choose the background color; you must add \usepackage{color} or \usepackage{xcolor}
basicstyle=\small, % the size of the fonts that are used for the code
breakatwhitespace=false,         % sets if automatic breaks should only happen at whitespace
breaklines=true,                 % sets automatic line breaking
captionpos=b,                    % sets the caption-position to bottom
commentstyle=\small\color{black},    % comment style
deletekeywords={...},            % if you want to delete keywords from the given language
escapeinside={\%*}{*)},          % if you want to add LaTeX within your code
extendedchars=true,              % lets you use non-ASCII characters; for 8-bits encodings only, does not work with UTF-8
frame=single,                    % adds a frame around the code
keepspaces=true,                 % keeps spaces in text, useful for keeping indentation of code (possibly needs columns=flexible)
keywordstyle=\color{mykeywordgray}\bfseries,       % keyword style
language=[Sharp]C,                 % the language of the code
morekeywords={*,Select,where,select,Write, from, in, orderby, IEnumerable, Where, OrderBy, FindIndex, List, Count, Insert, Remove},            % if you want to add more keywords to the set
numbers=left,                    % where to put the line-numbers; possible values are (none, left, right)
numbersep=10pt,                   % how far the line-numbers are from the code
numberstyle=\color{mykeywordgray}, % the style that is used for the line-numbers
rulecolor=\color{titlegray},         % if not set, the frame-color may be changed on line-breaks within not-black text (e.g. comments (green here))
showspaces=false,                % show spaces everywhere adding particular underscores; it overrides 'showstringspaces'
showstringspaces=false,          % underline spaces within strings only
showtabs=false,                  % show tabs within strings adding particular underscores
stepnumber=1,                    % the step between two line-numbers. If it's 1, each line will be numbered
stringstyle=\color{black},     % string literal style
tabsize=2,                       % sets default tabsize to 2 spaces
title=\lstname,                   % show the filename of files included with \lstinputlisting; also try caption instead of title
captionpos=t,
aboveskip=1\baselineskip,		% Platz über dem quellcode block
belowskip=1\baselineskip,			% Platz unter dem quellcode block
%morecomment=[il]{///}
}
% Settings fuer den Sourcecode ENDE

% Listings
\renewcommand{\lstlistlistingname}{Verzeichnis der Sourcecode Beispiele}
\renewcommand{\lstlistingname}{Sourcecode Beispiele}

% Autoren
\author{
Dominik Steffen, B. Sc.
}


% Titel
\title{Getting rid of VR Simulator Sickness}
\subtitle{Aufbau eines Best Practice Frameworks zur Integration der Oculus Rift in interaktiver Software als Anzeige und Eingabedevice unter der Berücksichtigung der Reduzierung der Simulator Sickness}
\parindent 0pt


%%%%%%%%%%%%%%%%%%%%%%%%%%%%%%%%%%%%%%%%%%%%%%%%%%%%%%%%%%%%%%%%%%%%%%%%%%%%%%%%
%	Commands START - Makros
%%%%%%%%%%%%%%%%%%%%%%%%%%%%%%%%%%%%%%%%%%%%%%%%%%%%%%%%%%%%%%%%%%%%%%%%%%%%%%%%
% C# makro OHNE space nach dem logo
\newcommand{\CS}{C\texttt{\#}}
% C# makro MIT space nach dem logo
\newcommand{\CSS}{C\texttt{\# }}
% C++ Logo
\newcommand{\CPP}{C\nolinebreak\hspace{-.05em}\raisebox{.4ex}{\tiny\bf +}\nolinebreak\hspace{-.10em}\raisebox{.4ex}{\tiny\bf +}}
% LINQ For Geometry
\newcommand{\LFG}{LINQ For Geometry}
% LINQ For Geometry mit Space
\newcommand{\LFGS}{LINQ For Geometry }
% LINQ mit spaces links und rechts
\newcommand{\LQ}{ LINQ }
% Generic zeichen <T>
\newcommand{\GT}{\textless T\textgreater}
\newcommand{\GTS}{\textless T\textgreater\space}
% Lambda Zeichen in C#
\newcommand{\LAM}{ =\textgreater\space}
% HES
\newcommand{\HES}{Half-Edge Datenstruktur }
%%%%%%%%%%%%%%%%%%%%%%%%%%%%%%%%%%%%%%%%%%%%%%%%%%%%%%%%%%%%%%%%%%%%%%%%%%%%%%%%
%	Commands ENDE
%%%%%%%%%%%%%%%%%%%%%%%%%%%%%%%%%%%%%%%%%%%%%%%%%%%%%%%%%%%%%%%%%%%%%%%%%%%%%%%%


%%%%%%%%%%%%%%%%%%%%%%%%%%%%%%%%%%%%%%%%%%%%%%%%%%%%%%%%%%%%%%%%%%%%%%%%%%%%%%%%
%	Unterstrichene Kapitelüberschriften START
%%%%%%%%%%%%%%%%%%%%%%%%%%%%%%%%%%%%%%%%%%%%%%%%%%%%%%%%%%%%%%%%%%%%%%%%%%%%%%%%
\newcommand*{\ORIGchapterheadendvskip}{}%
\let\ORIGchapterheadendvskip=\chapterheadendvskip
\renewcommand*{\chapterheadendvskip}{%
\ORIGchapterheadendvskip
{%
\setlength{\parskip}{0pt}%
\noindent\rule[3\baselineskip]{\linewidth}{1pt}\par
}%
}
%%%%%%%%%%%%%%%%%%%%%%%%%%%%%%%%%%%%%%%%%%%%%%%%%%%%%%%%%%%%%%%%%%%%%%%%%%%%%%%%
%	Unterstrichene Kapitelüberschriften ENDE
%%%%%%%%%%%%%%%%%%%%%%%%%%%%%%%%%%%%%%%%%%%%%%%%%%%%%%%%%%%%%%%%%%%%%%%%%%%%%%%%

%\newpage

\makeindex
\onehalfspacing
%\setuptoc{toc}{numbered}

\begin{document}
% Titelblatt START
\maketitle
%\addcontentsline{toc}{chapter}{Titelblatt}
%\includepdf[pages={1}]{../Bilder/deckblatt.pdf}
% Titelblatt ENDE

%\newpage
%\thispagestyle{empty}
%\mbox{}

% Inhaltsverzeichnis START
\begingroup
	%\clearpage
	%\addcontentsline{toc}{chapter}{Inhaltsverzeichnis} 
	\tableofcontents
	%\clearpage
	\thispagestyle{empty}
\endgroup
% Inhaltsverzeichnis ENDE
%\newpage
%\thispagestyle{empty}
%\mbox{}

%%%%%%%%%%%%%%%%%%%%%%%%%%%%%%%%%%%%%%%%%%%%%%%%%%%%%%%%%%%%%%%%%%%%%%%%%%%%%%%%
% Inhalt START
%%%%%%%%%%%%%%%%%%%%%%%%%%%%%%%%%%%%%%%%%%%%%%%%%%%%%%%%%%%%%%%%%%%%%%%%%%%%%%%%

% Passe Seitenzahlen wieder an START
\renewcommand*{\chapterpagestyle}{plain}
\pagestyle{plain}
\setcounter{page}{0}
% Passe Seitenzahlen wieder an ENDE

%%%%%%
%	Einführung / Einleitung START
%%%%%%
\chapter{Eine (technische) Einführung in die virtuelle Welt}

\section{Überblick}
Bei der Oculus Rift handelt es sich um ein Anzeige und Eingabedevice der Firma Oculus VR Inc. aus Californien. Die Oculus Rift, im folgenden kurz Rift genannt, ist ein Virtual Reality Headset einer völlig neuen Generation. Das Device wird vom Nutzer wie eine Skibrille getragen. Anwendungen welche speziell für die Rift entwickelt wurden bieten dem Nutzer eine immersive 3D Erfahrung. Genau an diesem Punkt greift dieses Projekt an. Eine Anwendung muss, um sie wirklich angenehm mit der Rift nutzen zu können, besondere Bedingungen erfüllen. Die Brille unterscheidet sich von normalen Stereodisplays durch einige besondere Eigenschaften. Sie verfügt über einen Monitor und zwei Linseneinsätze. Der Bildschirm, in der mitte halbiert, gibt zwei Bilder aus, eines für jedes Auge des Nutzers, er kann somit die virtuelle Welt in 3D erleben. Dabei beträgt das Sichtfeld des Nutzers, das Field of View (kurz. FOV) mehr als 90 Grad (horizontal). Die Brille verfügt, für das Bewegungstracking, über ein Gyroskop welches Bewegungen des Nutzers mitschneidet und diese über eine USB Kabel Verbindung an die auf einem Rechner laufende Applikation zurück gibt. So kann sich der Nutzer in der Virtuellen Welt bewegen, drehen und neigen.

Besonders durch den geringen Preis ist die Oculus Rift für Consumer interessanter als vergleichbare Geräte anderer Hersteller, die meist bei einem wesentlich geringeren FOV zu einem deutlich höheren Preis auf dem Markt sind. Ein in seiner Anwendung ähnliches System von Sony, technisch aber kaum vergleichbar da es wesentlich weniger Möglichkeiten für Spieler bietet, ist das HMZ-T3W zu einem Preis von ca. 1.299,00 Euro. Der spätere Verkaufspreis der Rift soll laut OculusVR in der geplanten Consumer Version nicht über 400 US Dollar liegen. Der Preis des aktuellen Development Kits der ersten Generation liegt bei 300 US Dollar was zum aktuellen Zeitpunkt in etwa 220,- Euro entspricht.

Anfang 2014, ist das VR Device nur als Entwicklerversion (Development Kit) verfügbar. Es kann aber von jedem Interessenten, ob in der Entwicklung tätig oder nicht, bestellt werden. Diese Version des Gerätes ist dazu gedacht an der Thematik der Virtuellen Realität zu forschen und Problembereiche der Softwareentwicklung für den VR Bereich zu identifizieren und zu beheben. Für Consumer ist das Produkt noch nicht gedacht. Es existiert noch so gut wie kaum eine Consumer taugliche Software. Bei der meisten Software handelt es sich um Demos die einen speziellen Sachverhalt oder ein bestimmtes Problem mit der Brille modellieren und versuchen es durch Softwareseitige Algorithmen und Designideen zu beheben.

\subsection{Ein neuer Prototyp}
Das Unternehmen Oculus VR arbeitet selbst stetig an einer Weiterentwicklung des Systems. Auf der Consumer Entertainment Show (CES) 2014 in Las Vegas USA, stellte das Unternehmen einen neuen Prototyp der Entwicklerversion vor. Dieser Crystal Cove genannte Prototyp bietet im Gegensatz zum aktuell erhältlichen Development Kit ein höher aufgelöstes Display und damit eine höher Aufgelöste Darstellung der Grafik. Weiterhin wurde eine Möglichkeit den Nutzer während der Nutzung der Rift im Raum zu tracken entwickelt. Diese Maßnahmen wurden von Oculus VR als Antwort auf die bei sehr vielen Rezipienten massiv auftretende Simulator Sickness entwickelt. Zum aktuellen Zeitpunkt ist dieser Prototyp für Entwickler leider noch nicht zugänglich weßhalb sich diese Arbeit mit einem dem Autor zur Verfügung stehenden Development Kit v1.0 beschäftigt.

\subsection{Eine Plattform zur Diskussion}
Die Webseite von Oculus VR bietet den Entwicklern und Forschungsteams, welche sich an vielen Universitäten und Hochschulen der ganzen Welt zusammen finden eine Plattform zum Erfahrungsaustausch und zur Diskussion der neusten Entwicklungen im Bereich Virtual Reality. Viele Einflüsse der Community finden ihren Weg in die weitere Entwicklung der Oculus Rift. Im November 2013 wurde ein von der Community angeregter Latency Tester durch Oculus VR umgesetzt und kann seit kurzem als Hardware Debugging Device geordert werden. Das Gerät unterstützt Entwickler bei der Messung der Verzögerungszeiten (Latenzen, eng. latency) welche bei der Messung und Verarbeitung der Head Tracking Daten entstehen. Eine Latenz von unter 20ms\footnote{} wird von OculusVR und vielen namhaften Entwicklern, unter anderem John Carmack, in der Gaming- und Interactive-Software Branche als Ziel angestrebt. Carmack ist aktuell CTO (Chief Technical Officer) von Oculus VR und hat für diesen Posten seine langjährige Stelle bei id Software in Dallas, Texas aufgegeben. Carmack war in den 90er Jahren maßgeblich an der Verbreitung von 3D Echtzeit Engines beteiligt. Viele Entwickler und Unternehmen sehen in der Oculus Rift die Zukunft des Gaming und einen wichtigen Schritt in der Weiterentwicklung der Branche.


\section{Über dieses Projekt}
Diese Arbeit möchte einen Teil zur aktuellen Entwicklung beitragen und beschäftig sich im Kern mit der technischen als auch design orientierten Reduzierung der aktuell noch massiv auftretenden Simulator Sickness. Ein Best Practice Framework, sowohl in Schriftlicher Form als auch als Software soll anderen interessierten Entwicklern einen einfachen Einstieg in die Entwicklung von VR Software ermöglichen und dabei die allgemeinen Fehler automatisch verhindern und aufklären. Dies ist ein wichtiger Schritt um zu einem Consumer Markt fähigen Produkt zu gelangen. Der Flop der, oft technisch schlecht umgesetzten, 3D Filme im Kino vor etwa zehn Jahren zeigte, dass mit neuen Techniken sehr sorgsam umgegangen werden muss und einiges an Forschung nötig ist um ein solches Produkt auf dem Massenmarkt zu etablieren. Es gibt sowohl auf technischer als auch auf theoretischer Ebene noch ungeklärte Fragen bezüglich der Implementierung von Software im Zusammenhang mit der Rift als Head Mounted Display (HMD).

Das Framework soll letzendlich eine Unterstützung für die Entwicklung neuer Produkte sein. Es soll helfen die massivsten Simulator Sickness Symptome zu vermeiden und ein angenehmes Nutzungserlebnis für den Rezipienten erschaffen. Zu den häufigsten Symptomen sind hier zu zählen:
\begin{itemize}
\item Ermüdung der Augen
\item Schwindlgefühle
\item Orientierungsverlust
\item Übelkeit
\end{itemize}

Diese Arbeit orientiert sich sowohl dem aktuell noch als Entwurf gekennzeichneten Best Practice Dokument von OculusVR sowohl als auch der Seite \url{http://developer.oculusvr.com/best-practices} welche sich im austausch mit Entwicklern und dem OculusVR eigenen Team der Problematik Simulator Sickness widmet. Zusätzlich werden im Laufe des Projektes verschiedene Software und Probanten Tests mit den Devices durchgeführt. Zur Zeit ist eine zukünftige Entwicklung des System schwer vorraus zu sagen. So kann es sein, dass während des Projektes der Crystal Cove Prototyp, oder ein noch weiter Entwickelter Typ für Entwickler zugänglich wird.

%%%%%%
%	Einführung / Einleitung ENDE
%%%%%%



%%%%%%
%	Hauptteil START
%%%%%%
\chapter{Probleme der virtuellen Welt und wie dieses Projekt versucht sie zu lösen}
\section{Beschreibung der Hardware und der technischen Spezifikationen}
Das Oculus Rift System, meist nur als die Oculus Rift oder als die oder das Rift bezeichnete System besteht aus folgenden Komponenten.
\begin{itemize}
\item Der Brille selbst
\item Dem USB Tracking Daten transport und Display Data transport Device
\item Einer Verbindung an eine Grafikkarte (GPU) eines Rechners über HDMI oder DVI Kabel
\item Einer Stromversorgung zum Daten und Display Device
\end{itemize}
Die Rift benötigt zur installation verschiedene Treiber welche jedoch über eine USB Verbindung auf einem Windows oder Mac System installiert werden oder je nach System bereits vorhanden sind. Diese Treiber werden zur Kommunikation des Systems mit dem Head Tracking Device der Brille genutzt und sind für die Verwendung des Systems essentiell. Das System kann über einen DVI oder HDMI Anschluss an einer Grafikkarte (GPU) betrieben werden. Es ist hierbei darauf zu achten, dass die GPU ausreichende Rechenkapazitäten beistzt, da sie für den Betrieb der Rift eine Rendering Szene zwei mal aus leicht verschiedenen Blickwinkeln rendern und verarbeiten muss.

\subsection{Das Display der Oculus Rift}
Das Display der aktuellen Generation des Development Kits unterstützt eine Auflösung von insgesamt 1280x800 Pixeln. Hierbei handelt es sich um ein 16:10 Format. Die Auflösung entspricht ungefähr einer HD Ready Auflösung im TV Format. Durch die Aufteilung des Displays auf zwei Augen halbiert sich die Pixelzahl für jedes Auge auf 640x800 Pixel. Das entspricht einem 4:5 Seitenverhältnis. Hierdurch wird es möglich ein Sichtfeld von etwa 90 Grad (horizontal) und 110 Grad (diagonal) zu erreichen. Der von Oculus VR auf der E3 und der CES 2014 vorgestellte Prototyp verfügt bereits über eine Full HD Auflösung von 1920x1080 Pixeln. Das Ziel von Oculus VR ist es eine Consumer Variante mindestens im Full HD Format anzubieten und wenn möglich sogar auf einen 4K Prototyp zu erhöhen.

\subsection{Die Linsen und ihre Notwendigkeit}
Das Display ist in einem Kunststoffgehäuse der Oculus Rift integriert und der Nutzer betrachtet das Display aus wenigen Zentimetern Entfernung. Durch die geringe Entfernung könnte der Nutzer seine Augen nicht auf das Display fikussieren und folglich auf dem Display nichts erkennen. Das Rift System ist aber mit einer Vorrichtung für zwei Linsen ausgestattet. Diese Linsen ermöglichen es, dass der Nutzer aus kurzer Entfernung ein riesiges Sichtfeld erlebt und sich seine Zirkularmuskel trotzdem entspannen können. Durch den Schliff der Linsen sind die Augen bei der Nutzung der Rift entspannt. Sie fixieren sich theoretisch auf eine Entfernung von mehr als 7 Metern was den Zirkularmuskel der Pupille entspannt. Durch die Verzerrung der Linsen ist es aber nötig das Bild bereits im Rechner über einen speziellen Grafik Shader mit einer Barrel distortion (Fassartigen Verzerrung) zu versehen. Hierdurch entzerren die Linsen das Bild wieder. Für Brillenträger werden beim Entwicklerkit Linsen mit verschiedener Sehstärke mitgeliefert. Diese Linsen sind durch einfaches einsetzen und drehen in das Rift System zu integrieren und können jederzeit ausgetauscht werden.

Es ist möglich die Entfernung des Displays zu den Linsen durch zwei Schrauben zu regeln. Dadurch bietet sich die Möglichkeit weitere Sehschwächen auszugleichen. Das System zeigt sich hier sehr variabel und ist gut an unterschiedliche Nutzertypen anzupassen.

\subsection{Head Tracking mit Hilfe von Gyrosensorsoren  und Beschleunigungssensoren}
Das Headtrackingsystem besteht aus einer Kombination eines 3 Achsen Gyrometers mit Beschleunigungssensoren. Ein Magnetometer unterstützt hierbei die Ausrichtung des gerenderten Views auf dem Bildschirm der Rift. Aktuell werden von den Bewegungssensoren nur Rotationen der Brille in verschiedene Richtungen registriert und in Fließkommawerten an den Treiber per USB Verbindung weiter gegeben. Bewegungen der Brille entlang der Achsen können aktuell noch nicht getracked werden. Der Crystal Cove Prototyp unterstützt ein "Positions Tracking" im Raum durch die Integration einer Webcam in den Kreislauf des Rift Systems. Diese Kamera wird vor dem Rezipienten aufgestellt und tracked die Translationen des Head Mounted Displays (HMD) im Raum.
\\

Die Sensoren der Rift arbeiten mit einer Abtastrate von 1000Hz pro Sekunde. Diese Abtastrate genügt um ein fast Fehlerfreies Erlebnis zu erschaffen. Aber auch hier bietet der neue Prototyp wieder eine Verbesserung. Durch geringere Latenzen beim Tracking wird der Immersive Eindruck des Systems noch verstärkt. Allerdings lassen sich damit noch keine Latenzen von unter 20ms erreichen. Durch ständige Optimierung und Vorhersagealgorithmen (prediction algorithms) der Kopfbewegungen des Nutzers verbessern sich die Latenzen jedoch zusehens. Ein Wert von 50ms wird zwar als ausreichend repsonsive bezeichnet, ist allerdings merkbar verzögert im Gegensatz zu  Latenzzeiten unterhalb von 20ms. Diese Latenzeiten werden von der Firma OculusVR als Richtwerte angegeben und sollen im Laufe dieser Arbeit kontrolliert und gegebenenfalls korrigiert werden. Der technische Teil der Arbeit soll insbesondere Wert darauf legen einen sehr Latenzarme Verarbeitung der Trackingpipeline ohne unnötigen Overhead zu implementieren um hier bereits einen massiven negativen Einfluss auf das Wohlbefinden des Rezipienten zu verhindern.


\section{Ein Überblick über Simulator Sickness und die Ursachen}
Simulator Sickness wird im Allgemeinen als "`symptoms of discomfort that arise from using simulated environments"' (vgl. OculusVR - OculusBestPractices.pdf, 2014) bezeichnet.
Zur Simulator Sickness zählen verschiedene Symptome welche bei vielen Nutzern zu ernsthaftem Unwohlsein führen können. Zu diesen Symptomen gehören unter anderem Schwindel, Übelkeit, Ermüdung der Augen, Verlust der Orientierung, Kopfschmerzen und weitere. Simulator Sickness entsteht aus einem Konflikt zwischen Sensorischer und Geistig erlebter Bewegung TODOZITAT. Es ist kaum vorher zu sagen ob ein Nutzer die genannten Symptome oder weitere erleben wird wenn er ein VR System oder die Oculus Rift im speziellen verwendet. Es ist auch nicht davon auszugehen, dass ein Nutzer der noch nie Simulator Sickness verspürte sie mit der Rift auch nicht verspüren wird. Simulator Sickness kann plötzlich auftreten und die Ausbildung einer Immunität ist zum jetztigen Kenntnisstand nicht möglich. Es ist allerdings in der Tat möglich den Körper auf die Verwendung von VR Systemen zu trainieren und so den Einfluss der Simulator Sickness auf den Körper über den Verlauf der Zeit zu reduzieren. Als Beispiel sollen an dieser Stelle die Erlebnisse des Autors angeführt werden. Zu Beginn der Arbeit mit der Rift konnten in etwa 15 Minuten Zeit in der Virtuellen Realität verbacht werden, wohingegen nach mehrfacher Nutzung die Zeiten aktuell auf bis zu zwei Stunden gesteigert werden konnten - ohne das massive  Eintreten von Simulator Sickness. Es ist jedoch anzuraten regelmässige Pausen einzulegen.

\section{Ansätze zur Reduzierung von Simulator Sickness in der technischen Umsetzung}
Simulator Sickness resultiert aus verschiedenen Faktoren während der Nutzung eines VR Systems. Die wichtigsten Faktoren hierbei sind die Bildwiederholrate, die Beschleunigung von Animationen und Bewegungen im dreidimensionalen Raum, die Auflösung des Anzeigedisplays als auch die Latenz der Datenübertragung zum Rechner und der Verarbeitung während des Trackings von Kopf Positionen und Rotationen. 

Um das Schwindelgefühl weiter einzudämmen sollten keine extremen Beschleunigungen ohne Vorwarnungen in der Interaktiven Software vorhanden sein. Hierzu zählen nicht interaktive Animations Sequenzen als auch vom Nutzer gesteuerte Bewegungen. Viele Nutzer reagieren sehr anfällig auf Bewegungsanimationen welche plötzliche Bewegungscycles enthalten. Es hat sich laut Oculus VR auch herausgestellt, dass Spieler sich mit einer Laufgeschwindigkeit von etwa 1,6 Metern pro Sekunde am besten in der Virtuellen Realität bewegen können. Ein solcher Vergleichswert soll während dieser Arbeit für verschiedene Bewegungsabläufe entwickelt und getestet werden.

Die Auflösung des Displays ist aktuell eines der größten Probleme der Simulator Sickness im Zusammenhang mit der Oculus Rift.  Eine Auflösung pro Auge mit 640x800 Pixeln ist für heutige Spiele und Spieler nicht mehr ausreichend. Im speziellen sind schmale Objekte wie z.B. dynamische Grashalme ein Problem, da durch die geringe Anzahl an Pixeln die feinen Animationen nicht in genügend Abstimmungen dargestellt werden. Durch diese Problematik springen schmale Objekte auf dem Display hin und her und erzeugen ein Flackern im Bild. Dieses Flackern löst beim Nutzer oft ein Unwohlsein bis hin zum Schwindelgefühl und Übelkeit aus. Es ist also darauf zu achten während der Entwicklung für das Devkit keine extrem feinen Strukturen zu verwenden. Die Größe des Interfaces sollte sich ebenso an der Auflösung orientieren. Wichtige Texte und Elemente der Anzeige und Informationen für den Rezipienten müssen ausreichend groß dargestellt werden. Es ist wichtig Menüs und Texte besonders bei Portierungen für die Rift neu zu gestalten. Einfache Portierungen sind für den Nutzer nur schwer zu ertragen da Texte auch zu den feineren Strukturen in einer interaktiven Software zählen.

\subsection{Frames per Second - Bildwiederholrate im Zusammenhang mit Motion Sickness}
Die Bildwiederholrate sollte nach einer Empfehlung von OculusVR auf dem Oculus internen Anzeigedisplay über 60 Bilder pro Sekunde betragen. Diese Frames per Second (FPS) Zahlen verhindern auf dem Oculus Rift fähigen Display ein stottern und verschmieren der gerenderten Bilder und unterstützen so die flüssigere Darstellung der Inhalte. Durch eine flüssigere Darstellung lässt sich die Ermüdung der Augen als auch das Schwindelgefühl bekämpfen. Eine flüssigere Darstellung der Software lässt sich auf verschiedene Arten erreichen. In diesem Projekt soll hierauf nur im Konzeptionellen Teil eingegangen werden. Der Entwickler sollte auf jeden Fall immer die Framerate der Anwendung im Blick haben und wenn nötig auf Dekorationseffekte wie Partikeleffekte und Shader verzichten wenn sie die Framerate auf angestrebten Systemen unter 60 FPS droppen lässt.

\subsection{Rendering und der Screen Door Effekt - auch Fliegengittereffekt}
Um mit der aktuellen Version der Oculus Rift ausreichend gute Ergebnisse im Bereich des Rendering und der Grafikpräsentation zu erzielen ist es anzuraten für den Renderer der Software ein Supersampling und Anti-Aliasing System zu verwenden. Das bedeutet die Auflösung des Renderbuffers mit einem vielfachen der tatsächlichen Ausgabeauflösung zu initialisieren. Hierdurch wirkt das Bild wesentlich ruhiger als wenn es in einer natürlichen Auflösung berechnet wird. Aktuelle Software wie Battlefield4 bieten Supersamling auch ohne Rift integration an um das Bild im Allgemeinen ruhiger wirken zu lassen und dem Nutzer ein Kantenfreies Erlebnis zu bieten. 

\subsubsection{Der Screen Door Effekt - Ein temporäres Hardware Problem}
Beim Screen Door Effekt handelt es sich um ein Phänomen das besonders durch die Verwendung der Linsen im Rift System hervor tritt. Zuletzt wurde das Problem vor einigen Jahren bei Monitoren mit sehr geringen Auflösungen aber großer Display Diagonale wahr genommen. Das Problem beschreibt, dass der Nutzer die Pixel eines Anzeigegerätes genau ausmachen kann. Durch die hohen Auflösungen der heutigen Anzeigegeräte hat sich diese Problematik eigentlich über die Jahre selbst beseitigt. Durch die vergrößerung der Linsen in der Oculus Rift werden die Pixel des Displays allerdings so stark vergrößert dass der Nutzer sie ohne Anstrengung von einander unterscheiden kann. Ein HD Display in der Oculus Rift wie im aktuellen Crystal Cove Prototypen zu sehen verbesser diesen Zustand schon ausreichend. Die Verwendung eines 4K Displays als Anzeigedevice würde das Problem gänzlich lösen. Diese Displays sind in der Herstellung zum aktuellen Zeitpunkt aber noch um einiges teurer als ein einfaches FullHD Display. Ob ein solches Display es in die erste Consumer Version der Oculus Rift schafft ist aktuell noch nicht abzusehen. 

In diesem Projekt besteht also nicht die Möglichkeit diesen Bereich für die Nutzer angenehmer zu gestalten, da sich das Phänomen des Screen Door Effektes auf der Hardware Ebene der Rift manifestiert und sich dieses Projekt lediglich mit Software Problemlösungen beschäftigt.

\section{Ansätze zur Reduzierung von Simulator Sickness im konzeptionellen Bereich}
Grundsätzlich gibt es verschiedene Ansätze Simulator Sickness konzeptionell zu verhindern. Zuerst ist es möglich, dass der Rezipient während der Nutzung der Interaktiven Software versucht durch Bewegungen seines Körpers, oder die Nutzung eines Systems mit haptischem Feedback wie z.B. Force Feedback Systemen ein Aufkommen von Simulator Sickness verhindert oder zumindest die Chance Symptome zu erleiden drastisch zu senkt. Aktuell ist das "`Mitbewegen"' des Körpers eine gute und vor allem günstige Möglichkeit während des Entwickelns die Testzeiten in der Virtuellen Realität zu verlängern und das Erlebnis trotzdem immersiver zu gestalten. Dieser Aspekt hat allerdings seine Grenzen. Es ist grundsätzlich zu vermeiden den Nutzer dazu zu animieren während einer Anwendung zu viel Bewegung auszuführen, da durch die sehr eingeschränkte Sicht mit der Brille nach Außen Verletzungsgefahr besteht.

Während der Entwicklung einer immersiven interaktiven Anwendung ist darauf zu achten, dass das Modul welches die Steuerung des Charakters in der Anwendung realisiert (Character Controller) sauber implementiert ist und für verschiedene Nutzer Typen verschiedene Einstellungsmöglichkeiten bietet. Insbesondere eine Kalibrierung der Inter Ocular Distance (IOD), dem Abstand von Pupille zu Pupille, ist für das erträgliche Nutzen einer solchen Anwendung von entscheidender Bedeutung. Eine gute Anwendung unterscheidet sich hier von einer schlechten durch eine für den Nutzer einfach durchzuführenden Dokumentierten Kalibration des Systems. OculusVR bietet hier ein grundsätzliches Tool zur Kalibrierung des Systems an. Dieses System arbeitet aber nur auf Treiber Ebene und ist kein Ersatz für eine saubere Umsetzung der Kalibrierung in der Anwendung.
Weiterhin gehört zur Umsetzung dieses Frameworks ein für andere Entwickler frei konfigurierbarer Controller der sowohl Bewegungsgeschwindigkeiten als auch verschiedene Kameraeinstellungen zulässt.

Es ist ebenfalls angedacht dem Nutzer verschiedenste API Schnittstellen und Eingabemöglichkeiten zur Entwicklung an die Hand zu geben. Konrekt sind hier verschiedene Typen von Eingabedevices wie z.B. Gamepads, klassisch Tastatur und Maus so wie Bewegungssteuerungen wie Kinect und vergleichbare Systeme gemeint. Ein Gamepad hat sich während verschiedener Tests unter den Entwicklern dieses Projekts als beste Möglichkeit der Eingabe heraus gestellt. Dieser Ansatz wird während der Forschung weiter betrachtet und optimiert.


\section{Mögliche Inhalte eines Frameworks}
Das Framework soll eine grundsätzliche Implementierung der wichtigsten Bestandteile eines Oculus Rift fähigen Projekts enthalten. Hiezu zählen ein Character Controller welcher über verschiedenste Input Methoden angesprochen werden kann. Des weiteren müssen für die erfolgreiche Implementierung des Charakter Controllers verschiedene Kamera Funktionen und Module erstellt werden. Wichtig sind hier vor allem einige vorkonfigurierte Werte an welchen sich andere Entwickler orientieren können. Ein Einheitliches Standard Interface auf Software Seite und in den Konfigurationswerten ist hier unumgänglich. Das gesamte Framework soll in die Furtwangen Entertainment and Simulation Engine der Hochschule Furtwangen (kurz FUSEE) integriert werden. Die FUSEE Engine ist hier ausreichend offen um die angestrebten Ziele umzusetzen und viele Studierende der Hochschule verwenden diese Engine zur Forschung und Lehre.

\subsection{Cameras}
Ein Kamera Modul soll verschiedene Funktionen unterstützen. Das Field of View muss genau  so wie eine relative Entfernunsberechnung zwischen den Kameras realisiert werden um die IOD des Systems exakt einzustellen. Zusätzlich muss für die Kameras und die Projektionsmatrix derselben ein Barrel Shader geschrieben werden welcher in Ansätzen bereits von OculusVR geliefert wird. Der Shader kann an verschiedenen Stellen noch verfeinert werden.
\subsection{Input Methoden}
Als wichtigste Eingabemethode hat sich während verschiedener Tests ein Gamepad herausgestellt. Hierfür müssen Schnittstellen geschaffen werden. FUSEE bietet hier die Möglichkeit einige XInput fähige Geräte über eine \CSS API und DirectX anzusprechen. 
\subsection{Character Controller}
Der Character Controller selbst muss eine Verbindung zwischen den anderen Implementierten Modulen herstellen. Zusätzlich wird er ein gewisses Maß an Motion prediction (Vorhersage von Bewegungen) berechnen können.
\subsection{Richtwerte}
Die Richtwerte für die Konfigurationen einer Oculus Rift fähigen Anwendung sollen in Config Dateien mit einem XML artigen Schema abgelegt werden. Diese Dateien sollen zur Laufzeit in das Softwaresystem geladen werden können. So kann ein Nutzer die Werte zur Laufzeit an seine aktuellen Ansprüche anpassen. Die Anwendung selbst und vor allem die Rendepipeline soll darauf ausgelegt werden eine Bildwiederholrate von 60 Frames pro Sekunde zu erreichen. Dazu zählt, dass eine Auflösung von 640x800 für jedes Auge. In stärkeren Systemen kann ein vielfaches der Basisauflösung für Supersampling Berechnung verwendet werden um die Qualität der Ausgabe zu verbessern.

\section{Entwicklungsprozess - Agile Software Entwicklung}
Die Entwicklung erfolgt nach dem Prozess der agilen Softwareentwicklung und dem Scrum System. Dieses System wird hier gewählt damit ausreichend schnell auf Änderungen an der Treiber Software der Oculus Rift und anderen Software Systemen reagiert werden kann. Für die FUSEE Engine existiert aktuell nur eine sehr grundsätzliche Implementierung der Oculus Rift als Anzeigegerät. Die Trackingdaten können aktuell noch nicht aus dem Treiber ausgelesen werden. Hier sind also zuerst Schnittstellen zwischen dem OculusVR Treiber und der Game Engine zu schaffen. 

\section{Rezipienten Tests}
Zur überprüfung der Ergebnisse wird das Projekt als OpenSource Projekt veröffentlicht. Es ist ebenfalls angedacht verschiedene Tests mit Rezipienten durchzuführen. Hierzu zählen Tests mit Consumern und einem fertigen Prototypen (Demo) einer Anwendung und zusätzlich Interviews mit verschiedenen Entwicklern der FUSEE Engine um die Ergonomie der Entwickler Tools zu testen.

Die Consumer Tests beziehen sich hier auf Tests mit mehreren Konfigurationsdateien zur Vermeidung der Simulator Sickness. Die Probanten werden stets zu ihrem Befinden befragt um so eine Auswirkung der verschiedenen Einstellungen feststellen zu können.

Entwickler Tests werden durchgeführt in dem das Projekt neben der normalen Veröffentlichung an der Hochschule ebenfalls im Forum von OculusVR weiteren Entwicklern vorgestellt wird. Es ist hier von Nöten eine ausreichend große Akzeptanz unter den Entwicklern zu erreichen, so dass durch Rückmeldungen Rückschlüsse auf die Qualität der Entwicklungs Module gezogen werden können. 
%%%%%%
%	Hauptteil ENDE
%%%%%%



%%%%%%
%	Schluss START
%%%%%%
\chapter{Der Versuch einer Vorraussage der kurzfristigen Entwicklungen in der virtuellen Welt}
OculusVR hat bis jetzt keine Angaben zu einem möglichen Veröffentlichungsdatum der Oculus Rift gemacht. Es wird in der Branche jedoch angenommen, dass das System spätestens im vierten Quartal 2014 veröffentlicht werden soll. Aufgrund der schnellen Entwicklungen im Virtual Reality Bereich sind bis zur Veröffentlichung des Systems sicher noch einige Prototypen der Rift zu erwarten. Mit einem 4K System wird in der Branche noch - zumindest als Prototyp - vor dem Datum der Veröffentlichung gerechnet.

Viele Entwickler überdenken gerade eine Implementierung der Rift in ihre Software. Das Unternehmen CCP, verantwortlich für das Online Spiel EVE-Online hat am 05.02.2014 bekannt gegeben Zwecks der Entwicklung und Veröffentlichung von EVE Valkyrie (einem Oculus exclusiven Spiel) eng mit OculusVR zusammen zu arbeiten. \footnote{http://www.oculusvr.com/blog/eve-valkyrie-open-source-hardware-and-the-best-practices-guide} OculusVR agieren hierbei als Publisher für das erste Rift Exklusive Spiel eines großen europäischen Studios. Unternehmen wie Gaijin Entertainment haben bereits eine Rift Unterstützung in ihr erfolgreiches Online Spiel "`War Thunder"'\footnote{http://warthunder.com/de} als Beta Test Modul integriert. Valve Software, verantwortlich für die Online Plattform Steam und Spiele wie Half-Life und Counterstrike hat bereits eine Oculus Rift integration in den Shop und viele ihrer eigenen Spiele implementiert. Die Source Engine von Valve ist eine Engine mit einer zum aktuellen Zeitpunkt sehr ausgefeilten Rift Unterstützung. Auf der Steam Online Plattform gibt es aktuell sogar eine eigene Sektion mit Spielen (meist Indie Spiele (d.h. in Selbstveröffentlichung ohne Unterstützung durch Publisher) die für die Oculus Rift entwickelt wurden oder zumindest eine Grundsätzliche Implementierung der Funktionen des Systems bieten.

Durch das Auftreten dieser Unternehmen zeichnet sich ein wahrer Hype der VR Brille auf den Gaming und Entertainment Markt ab. Dieses Projekt möchte einen Beitrag leisten um einen neuen Abschnitt der Softwareentwicklung und Gestaltung mit all seinen Möglichkeiten zu betreten.
%%%%%%
%	Schluss ENDE
%%%%%%


%%%%%%%%%%%%%%%%%%%%%%%%%%%%%%%%%%%%%%%%%%%%%%%%%%%%%%%%%%%%%%%%%%%%%%%%%%%%%%%%
% Inhalt ENDE
%%%%%%%%%%%%%%%%%%%%%%%%%%%%%%%%%%%%%%%%%%%%%%%%%%%%%%%%%%%%%%%%%%%%%%%%%%%%%%%%
\part*{Anhang}


%%%%%%%%%%%%%%%%%%%%%%%%%%%%%%%%%%%%%%%%%%%%%%%%%%%%%%%%%%%%%%%%%%%%%%%%%%%%%%%%
% Source Code Verzeichnis START
%%%%%%%%%%%%%%%%%%%%%%%%%%%%%%%%%%%%%%%%%%%%%%%%%%%%%%%%%%%%%%%%%%%%%%%%%%%%%%%%
%\lstlistoflistings
%%%%%%%%%%%%%%%%%%%%%%%%%%%%%%%%%%%%%%%%%%%%%%%%%%%%%%%%%%%%%%%%%%%%%%%%%%%%%%%%
% Source Code Verzeichnis ENDE
%%%%%%%%%%%%%%%%%%%%%%%%%%%%%%%%%%%%%%%%%%%%%%%%%%%%%%%%%%%%%%%%%%%%%%%%%%%%%%%%

%%%%%%%%%%%%%%%%%%%%%%%%%%%%%%%%%%%%%%%%%%%%%%%%%%%%%%%%%%%%%%%%%%%%%%%%%%%%%%%%
% Tabellen Verzeichnis START
%%%%%%%%%%%%%%%%%%%%%%%%%%%%%%%%%%%%%%%%%%%%%%%%%%%%%%%%%%%%%%%%%%%%%%%%%%%%%%%%
%\listoftables
%%%%%%%%%%%%%%%%%%%%%%%%%%%%%%%%%%%%%%%%%%%%%%%%%%%%%%%%%%%%%%%%%%%%%%%%%%%%%%%%
% Tabellen Verzeichnis ENDE
%%%%%%%%%%%%%%%%%%%%%%%%%%%%%%%%%%%%%%%%%%%%%%%%%%%%%%%%%%%%%%%%%%%%%%%%%%%%%%%%

%%%%%%%%%%%%%%%%%%%%%%%%%%%%%%%%%%%%%%%%%%%%%%%%%%%%%%%%%%%%%%%%%%%%%%%%%%%%%%%%
% Abbildungsverzeichnis START
%%%%%%%%%%%%%%%%%%%%%%%%%%%%%%%%%%%%%%%%%%%%%%%%%%%%%%%%%%%%%%%%%%%%%%%%%%%%%%%%
%\listoffigures
%%%%%%%%%%%%%%%%%%%%%%%%%%%%%%%%%%%%%%%%%%%%%%%%%%%%%%%%%%%%%%%%%%%%%%%%%%%%%%%%
% Abbildungsverzeichnis ENDE
%%%%%%%%%%%%%%%%%%%%%%%%%%%%%%%%%%%%%%%%%%%%%%%%%%%%%%%%%%%%%%%%%%%%%%%%%%%%%%%%

%%%%%%%%%%%%%%%%%%%%%%%%%%%%%%%%%%%%%%%%%%%%%%%%%%%%%%%%%%%%%%%%%%%%%%%%%%%%%%%%
% Bilbiographie START
%%%%%%%%%%%%%%%%%%%%%%%%%%%%%%%%%%%%%%%%%%%%%%%%%%%%%%%%%%%%%%%%%%%%%%%%%%%%%%%%
\nocite{*}
\bibliography{Citavi}
\addcontentsline{toc}{chapter}{Literaturverzeichnis}
\newpage
%%%%%%%%%%%%%%%%%%%%%%%%%%%%%%%%%%%%%%%%%%%%%%%%%%%%%%%%%%%%%%%%%%%%%%%%%%%%%%%%
% Bilbiographie ENDE
%%%%%%%%%%%%%%%%%%%%%%%%%%%%%%%%%%%%%%%%%%%%%%%%%%%%%%%%%%%%%%%%%%%%%%%%%%%%%%%%

%%%%%%%%%%%%%%%%%%%%%%%%%%%%%%%%%%%%%%%%%%%%%%%%%%%%%%%%%%%%%%%%%%%%%%%%%%%%%%%%
% UML START
%%%%%%%%%%%%%%%%%%%%%%%%%%%%%%%%%%%%%%%%%%%%%%%%%%%%%%%%%%%%%%%%%%%%%%%%%%%%%%%%

%%%%%%%%%%%%%%%%%%%%%%%%%%%%%%%%%%%%%%%%%%%%%%%%%%%%%%%%%%%%%%%%%%%%%%%%%%%%%%%%
% UML ENDE
%%%%%%%%%%%%%%%%%%%%%%%%%%%%%%%%%%%%%%%%%%%%%%%%%%%%%%%%%%%%%%%%%%%%%%%%%%%%%%%%
\end{document}